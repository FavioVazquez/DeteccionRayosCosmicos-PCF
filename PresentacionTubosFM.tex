\documentclass[a4paper,10pt]{beamer}
\usepackage[utf8]{inputenc}
\usepackage{color}
\usepackage{colortbl}
\usepackage{xcolor}
\usepackage{caption}
\usepackage{ragged2e}
\usepackage{hyperref}
\usepackage{marvosym}
\renewcommand{\figurename}{Figura}
\usetheme{Warsaw}


\logo{\includegraphics[scale=0.06]{logoUNAM}}
\begin{document}

\begin{frame}
\title{Tubos fotomultiplicadores y fotodiodos}
\author{Favio Vázquez}
\date{$1^{ro}$ de septiembre de 2015}

\Large{Láminas disponibles en }
\maketitle
\end{frame}

\section[\'Indice]{}
\frame{

\Large{\'Indice}

\tableofcontents

}

\section{Introducción}
\begin{frame}[allowframebreaks]{Introducción}
\begin{justify}
 El uso masivo de la cuenta de centelladores en la detección y espectroscopia sería 
 imposible sin la disponibilidad de dispositivos que nos permitiera convertir la 
 salida de luz débil de un pulso de centellador, en una señal eléctrica medible. 
 
 \vspace{.3cm}
 
 Los tubos fotomultiplicadores (FM) cumplen con esta tarea muy bien, convirtiendo señales 
 de luz que constan típicamente de no más que unos cientos de fotones, en un pulso 
 de corriente utilizable sin añadir una gran cantidad de ruido aleatorio a la señal.
 
\framebreak

 Existe una gran variedad comercial de estos tubos sensibles a diversas longitudes de onda,
 ultravioleta, luz visible, cercana a la infrarroja y otras del espectro electromagnético.
 
 \vspace{.3cm}
 
 \textbf{Usos}:
 
 \begin{center}
   \includegraphics[scale=0.2]{fig1}
 \end{center}

 
 \end{justify}
\end{frame}

\section{Estructura simplificada de un Tubo FM}
\begin{frame}[allowframebreaks]{Estructura simplificada de un Tubo FM}
 
 \begin{columns}[c]
  \column{1in}
  \begin{figure}
  \center
   \includegraphics[scale=0.25]{fig2}
   \caption{Elementos básicos de un tubo FM}
  \end{figure}

  \column{1in}
  \begin{justify}
   
   Usualmente 
   
  \end{justify}

 \end{columns} 
 
\end{frame}



\end{document}
