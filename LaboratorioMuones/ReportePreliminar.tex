\documentclass[a4paper,10pt]{article}
\usepackage[utf8]{inputenc}
\usepackage[spanish]{babel}
\usepackage[affil-it]{authblk}
\usepackage{enumerate}
\usepackage{graphicx}
\usepackage{hyperref}
\usepackage{amsmath}
\usepackage{amssymb}
\usepackage{cancel}
\usepackage[usenames, dvipsnames]{color}
\usepackage{tikz}
\usepackage[labelfont=bf]{caption}
\usepackage{subcaption} %Multiple images
\usepackage{multicol} % Multiple columns
\usepackage{float}
\usepackage{cleveref}
 \usepackage{relsize} % bigger math symbols
\usepackage[margin=1.4in]{geometry}
\usepackage[titletoc,toc,title]{appendix}
\usepackage{enumitem}
\usepackage{etoolbox}
\usetikzlibrary{calc}
\numberwithin{equation}{section}

% Circled words
\newcommand{\circled}[2][]{%
  \tikz[baseline=(char.base)]{%
    \node[shape = circle, draw, inner sep = 1pt]
    (char) {\phantom{\ifblank{#1}{#2}{#1}}};%
    \node at (char.center) {\makebox[0pt][c]{#2}};}}
\robustify{\circled}

%Appendices in spanish
\renewcommand{\appendixname}{Ap\'endices}
\renewcommand{\appendixtocname}{Ap\'endices}
\renewcommand{\appendixpagename}{Ap\'endices}

%Zero delimiter
\newcommand{\zerodel}{.\kern-\nulldelimiterspace}

%Columns separation
\setlength{\columnsep}{1cm}

%Indentation
\setlength{\parindent}{0ex}

%Multiple References

\crefrangelabelformat{equation}{(#3#1#4--#5\crefstripprefix{#1}{#2}#6)}

\usepackage{xparse}

%Boxes

\newcommand*{\boxcolor}{blue}
\makeatletter
\renewcommand{\boxed}[1]{\textcolor{\boxcolor}{%
\tikz[baseline={([yshift=-1ex]current bounding box.center)}] \node [rectangle, minimum width=1ex,rounded corners,draw] {\normalcolor\m@th$\displaystyle#1$};}}
 \makeatother

%Constantes
\newcommand{\euler}{\mathrm{e}}
\newcommand{\im}{i}

% Definición de las secciones y su numeración

\makeatletter
\def\@seccntformat#1{%
  \expandafter\ifx\csname c@#1\endcsname\c@section\else
  \csname the#1\endcsname\quad
  \fi}
\makeatother

%opening
\title{{\huge Reporte Preliminar} \\
\vspace{.2cm}
\large Laboratorio Avanzado - Detección de Rayos Cósmicos}
\author{Favio Vázquez\thanks{Correo: favio.vazquezp@gmail.com}\affil{Instituto de Ciencias Nucleares. Universidad Nacional Autónoma de México.},
Susana Marín\thanks{Correo: susyma3005@gmail.com}\affil{Instituto de Química. Universidad Nacional Autónoma de México.}}
\date{}

\begin{document}

\makeatletter
\def\@maketitle{%
  \newpage
  \null
  \vskip 2em%
  \begin{center}%
  \let \footnote \thanks
    {\Large\bfseries \@title \par}%
    \vskip 1.5em%
    {\normalsize
      \lineskip .5em%
      \begin{tabular}[t]{c}%
        \@author
      \end{tabular}\par}%
    \vskip 1em%
    {\normalsize \@date}%
  \end{center}%
  \par
  \vskip 1.5em}
\makeatother

\maketitle

\section{Introducción}

\section{Marco Teórico}

\section{Metodología Experimental}

\subsection{Materiales utilizados}

A continuación se listan los materiales que fueron utilizados para las distintas mediciones:

\begin{itemize}
 \item Dos paletas centelladoras.
 \item Osciloscopio.
 \item Cables de 1, 3, 5, 10, 16 ns.
 \item Convertidores para cables.
 \item Fuente de alto voltaje.
 \item Placas de plomo.
 \item Flexómetro.
 \item Módulos de alto voltaje, temporización, discriminador y unidad lógica AND.
 \item Soporte de ángulo variable.
\end{itemize}


\subsection{Medición del punto de operación}



Se utilizó un arreglo de centelladores, llamado arreglo de coincidencias, con el cual se desea medir el punto de operación del sistema, y el rango de operación del mismo, 
lo cual nos servirá para mediciones posteriores. A continuación se describen los 
pasos realizados para hacer la primera parte del diseño experimental.

\vspace{.3cm}

Pasos:

\begin{itemize}
 \item Se activan las paletas y se ponen para hacer coincidencia. Se conectan las paletas a la fuente de alto voltaje y 
 mediante el programa HyperTerminal se fija un voltaje inicial y se fue aumentando el voltaje lentamente. Las paletas 
 se colocan para hacer coincidencia una sobre la otra y se fijan a la mesa.
 \item Posteriormente se conectan las paletas al osciloscopio con cables de 10 ns  para comprobar la coincidencia, para esto se deben ajustar 
 las escalas de voltaje y temporal del osciloscopio. La escala temporal se fija al rededor de 80 ns y 
 el voltaje en 10 mV.
 \item Se conectan las paletas al discriminador con cables de la misma longitud, para evitar desfases en la señal.
 \item Se ajusta el voltaje del umbral (threshold) a 13 mV, y se revisa que exista coincidencia conectando cables de 
 16 ns al osciloscopio.
 \item Se conectan los canales correspondientes y se procede a tomar nota de las cuentas de partículas que llegan al sistema
 en coincidencia en un tiempo determinado, que fue de 5 minutos para cada voltaje.
 \item Estas mediciones se repiten para cada voltaje y así obtener una estadística del proceso de detección, se hace una tabla y 
 se grafican los valores.
\end{itemize}


\subsection{Medición de la distribución angular de los muones}

Esta medición se hizo para medir la distribución angular de los muones y determinar 
el porcentaje de los mismos que llegan desde todo lugar.

\vspace{.3cm}

Se utilizó el soporte de ángulo variable, se fijó una de las paletas en la parte 
inferior del soporte y la otra paleta en la parte superior, habiendo una distancia 
entre ellos dos de 7 cm. Se conectó el sistema y se comprobó la coincidencia
de la misma manera que en la sección anterior. Se fue girando el dispositivo cada 
10 grados, partiendo de $0^\circ$ hasta $90^\circ$, y se anotó el número de cuentas 
de muones para cada ángulo en un tiempo de 5 minutos por medición, en la sección
de resultados se encuentran estos datos tabulados y graficados.

\section{Resultados}

\subsection{Medición del punto de operación}

Se realizaron tres mediciones debido a que los errores en las primeras dos fueron muy altos, 
ya que habían problemas con el cableado, y mucho ruido desde la fuente de la alto voltaje,
y en el sistema en general. La medición que fue tomada en cuenta para calcular el 
punto y rango de operación fue la tercera, que consistió en 5 mediciones para cada voltaje 
partiendo de 650 V a 850 V, subiendo de el voltaje de 20 en 20. 


\begin{table}[H]
\centering
\caption{Mediciones a distintos voltajes para el flujo de muones}
\label{my-label}
\resizebox{\linewidth}{!}{%
\begin{tabular}{|l|l|l|l|l|l|l|l|}
\hline
Voltaje & Medicion1 & Medicion2 & Medicion3 & Medicion4 & Medicion5 & Promedio & Error \\ \hline
650 & 90 & 89 & 91 & 104 & 90 & 92.8 & 6.3 \\ \hline
670 & 133 & 113 & 130 & 135 & 123 & 126.8 & 8.9 \\ \hline
690 & 135 & 138 & 147 & 146 & 133 & 139.8 & 6.3 \\ \hline
710 & 131 & 131 & 159 & 152 & 142 & 143.0 & 12.5 \\ \hline
730 & 153 & 142 & 157 & 153 & 170 & 155.0 & 10.0 \\ \hline
750 & 148 & 139 & 138 & 146 & 149 & 144.0 & 5.1 \\ \hline
770 & 139 & 146 & 168 & 159 & 159 & 154.2 & 11.5 \\ \hline
790 & 143 & 148 & 168 & 171 & 165 & 159.0 & 12.6 \\ \hline
810 & 144 & 140 & 162 & 177 & 178 & 160.2 & 17.8 \\ \hline
830 & 178 & 166 & 192 & 173 & 204 & 182.6 & 15.2 \\ \hline
850 & 180 & 195 & 197 & 199 & 193 & 192.8 & 7.4 \\ \hline
\end{tabular}}
\end{table}

De la anterior gráfica observamos que el rango de operación está entre 690 V y 770 V,
y por lo tanto tomamos el punto de operación como 740 V. 

\subsection{Medición de la distribución angular de los muones}

\section{Conclusiones}

\end{document}